\documentclass{article}
\usepackage{blindtext}
\usepackage[T1]{fontenc}
\usepackage[utf8]{inputenc}

\title{PVL 1 - theoretische Informatik}
\author{Sophia Köhler, 530976, Bachelor Informatik}
\date{16.10.2020}

\begin{document}

\maketitle

\section{}
Induktion besteht aus:
\begin{enumerate}
  \item Induktionsanfang.
  \item Induktionsschritt.
  	\begin{enumerate}
	  \item Induktionsbehauptung.
  	\item Induktionsbeweis.
  	\end{enumerate}
\end{enumerate}
\subsection{}
Induktion von \[ \sum_{k=1}^{n}{k} = \frac{(n+1)*n}{2} \]
\subsubsection{Induktionsanfang}

für $n=1$
\[ \sum_{k=1}^{1}{k} = \frac{(1+1)*1}{2} \]
\[ 1 = 1 \]
\newline
für $n=2$
\[ \sum_{k=1}^{2}{k} = \frac{(2+1)*2}{2} \]
\[ 1+2 = 3 \]
\newline
\clearpage
für $n=3$
\[ \sum_{k=1}^{3}{k} = \frac{(3+1)*3}{2} \]
\[ 1+2+3 = 6 \]
\newline
Induktionsanfang stimmt zunächst.
\subsubsection{Induktionsschritt}
\paragraph{Induktionsvoraussetzung}\mbox{}

Vorschrift gilt für: $k=n$ also \[1+2+3+n = \frac{(n+1)*n}{2} \]

\paragraph{Induktionsbehauptung}\mbox{}

Vorschrift gilt auch für: $k=n+1$ also \[1+2+3+...+(n+1) = \frac{(n+2)*(n+1)}{2} \]

\paragraph{Induktionsbeweis}\mbox{}

Aus der Induktionsbehauptung muss nun die Voraussetzung folgen.

es sollte also gelten: \[1+2+3+...+n+(n+1) = \frac{(n+1)*(n)}{2}+(n+1) = \frac{(n+2)*(n+1)}{2}\]
\[1+2+3+...+n+(n+1) = \frac{(n+1)*(n)}{2}+(n+1) \]
\[1+2+3+...+n+(n+1) = \frac{n^2+n}{2}+ \frac{2*n+2}{2} \]
\[1+2+3+...+n+(n+1) = \frac{n^2+3*n+2}{2}\]
pq Formel für das Lösen der rechten Seite: \[n_{1,2} = -\left(\frac{p}{2}\right) \pm \sqrt{ \left(\frac{p}{2}\right)^{2}-q} \]
 $n_1=-1$ und  $n_2=-2$
 \newline
nun folgt Umformung in Nullstellenform:
\[\frac{(n+2)*(n+1)}{2} \]
\newline
Somit ist gezeigt, dass Induktionsbehauptung aus der Induktionsvoraussetzung folgt.

\subsection{}
Induktion von \[ \sum_{k=1}^{n}{k^2} = \frac{(2*n+1)*(n+1)*n}{6} \]
\subsubsection{Induktionsanfang}

für $n=1$
\[ \sum_{k=1}^{1}{k^2} = \frac{(2*1+1)*(1+1)*1}{6} \]
\[ 1 = 1 \]
\newline
für $n=2$
\[ \sum_{k=1}^{2}{k^2} = \frac{(4*1+1)*(2+1)*2}{6} \]
\[ 1+4 = 5 \]
\newline

Induktionsanfang stimmt zunächst.
\subsubsection{Induktionsschritt}
\paragraph{Induktionsvoraussetzung}\mbox{}

Vorschrift gilt für: $k=n$ also \[1+2+9+n^2 =\frac{(2*n+1)*(n+1)*n}{6} \]

\paragraph{Induktionsbehauptung}\mbox{}

Vorschrift gilt auch für: $k=n+1$ also \[1+2+3+...+(n+1)^2 = \frac{(2*(n+1)+1)*((n+1)+1)*(n+1)}{6} \]
bzw.
\[1+2+3+...+(n+1)^2 = \frac{2*n^3+9*n^2+13*n+6}{6} \]
\clearpage
\paragraph{Induktionsbeweis}\mbox{}

Aus der Induktionsbehauptung muss nun die Voraussetzung folgen.

es sollte also gelten: \[1+2+3+...+(n+1)^2 = \frac{(2*n+1)*(n+1)*n}{6} + (n+1)^2= \frac{2*n^3+9*n^2+13*n+6}{6}\]
\[1+2+3+...+(n+1)^2 = \frac{(2*n+1)*(n+1)*n}{6}+(n+1)^2 \]
Ausmultiplizieren, binomische Formel und Erweitern:
\[1+2+3+...+(n+1)^2 = \frac{2*n^3+3*n^2+n}{6}+\frac{6*n^2+12*n+6}{6} \]
Addieren:
\[1+2+3+...+(n+1)^2 =  \frac{2*n^3+9*n^2+13*n+6}{6} \]
\newline
Somit ist gezeigt, dass Induktionsbehauptung aus der Induktionsvoraussetzung folgt.

\section{}
\subsection{}
Für welche n gilt:
\[n^2-1 \le \frac{1}{6}*n^3+\frac{1}{2}*n^2+\frac{1}{3}*n-1\]
Zunächst kann man auf beiden Seiten 1 addieren und ein n ausklammern:
\[n*n \le n*(\frac{1}{6}*n^2+\frac{1}{2}*n+\frac{1}{3})\]
Somit kennt man die erste Nullstelle, nämlich 0, da ein Term multipliziert mit null auch immer null ergibt.
Man rechnet nun weiter mit:
\[n \le \frac{1}{6}*n^2+\frac{1}{2}*n+\frac{1}{3}\]
Erweitern mit 6:
\[6*n \le n^2+3*n+2\]
\[0 \le n^2-3*n+2\]
p-q Formel für das Lösen der rechten Seite: \[n_{1,2} = -\left(\frac{p}{2}\right) \pm \sqrt{ \left(\frac{p}{2}\right)^{2}-q} \]
 $n_1=1$ und  $n_2=2$
 \newline
 Somit kennt man alle Nullstellen mit $n_1=0$,  $n_2=1$ und  $n_3=2$
 \newline
 Nun muss gezeigt werden, dass die Gleichung nur für alle  $0\le n \le 2$ gilt. Es folgt also Induktion.
 \subsection{}
Induktion von \[6*n \le n^2+3*n+2\]
\subsubsection{Induktionsanfang}

für $n=0$
\[0 \le 0\]
für $n=1$
\[6 \le 1+3+2\]
\[ 6 \le 6 \]
\newline
für $n=2$
\[12 \le 4+6+2\]
\[ 12 \le 12 \]
\newline

Induktionsanfang stimmt zunächst.
\subsubsection{Induktionsschritt}
\paragraph{Induktionsvoraussetzung}\mbox{}

Vorschrift gilt nur für:$n_1=0$,  $n_2=1$ und  $n_3=2$

\paragraph{Induktionsbehauptung}\mbox{}
Induktion über Gegenbeweis:
Vorschrift darf nicht gelten für: $n+1$ also Induktion von \[6*(n+1) \le (n+1)^2+3*(n+1)+2\]
\subsection{}
Für welche n gilt:
\[2*n*log_2n-n+1 \geq 3*n-3 \]
\[2*n*log_2n-n+4 \geq 3*n \]
\[2*n*log_2n+4 \geq 4*n \]
\[log_2n \geq 2 - \frac{2}{n} \]
Kann nicht weiter nach n aufgelöst werden. Da der Logarithmus nicht kleiner als null werden kann, kennt man bereits eine
mögliche untere Grenze. Durch probieren ergibt sich:
\newline
für $n=1$ \[4 \geq 4 \]
für $n=2$ \[1 \geq 1 \]
für $n=3$ \[1.585 \geq  1.\overline{3} \]
für $n=4$ \[2 \geq  1.5 \]
da gilt :\[\lim_{n\to\infty}\left(log_2n\right)=\infty\]
und 
:\[\lim_{n\to\infty}\left(2 - \frac{2}{n}\right)=2 \] 
weil $\frac{2}{n}$ eine monotone Nullfolge ist, kann man mit Sicherheit sagen, dass man ab n = 4 bewiesen hat, dass der Logarithmus immer größer ist als die rechte Seite. Da der Grenzwert 2 ist, der Logarithmus aber gegen $\infty$ konvergiert, ist die Ungleichung also gegeben für \[ 0 \textless n \le \infty \]
\section{}
\[V(n) = 2* V\left(\frac{n}{2}\right)+ b*n +c \] mit V(1) = 1 und $n \geq 2$
\newline
Lösen der Rekursionsgleichung zunächst durch Ausprobieren.
$n = 2^k$ : \[V(2^k) = 2* V\left(2^{k-1}\right)+  2^k*b +c \] 
\[V(2^k) = 2* \left(2*V\left(2^{k-2}\right)+2^{k-1}*b+c_2\right )+  2^k*b +c_1 \] 
\[V(2^k) = 2^2*V\left(2^{k-2}\right)+2*2^{k-1}*b+2*c_2+  2^k*b +c_1 \] 
\[V(2^k) = 2^2*\left(2*V\left(2^{k-3}\right)+2^{k-2}*b+c_3\right)+2*2^{k-1}*b+2*c_2+  2^k*b +c_1 \] 
\[V(2^k) = 2^3*V\left(2^{k-3}\right)+2^2*2^{k-2}*b+2^2*c_3+2*2^{k-1}*b+2*c_2+  2^k*b +c_1 \] 
\[V(2^k) = 2^3*V\left(2^{k-3}\right)+2^{k}*3*b+(2^2*c_3+2^1*c_2+ 2^0*c_1) \] 
\newline
c kann ausgeklammert werden, da das c durch die Rekursion immer den gleichen Wert hat.
\[V(2^k) = 2^j*V\left(2^{k-j}\right)+2^{k}*j*b+ c*\sum_{i=0}^{j-1}{2^i} \] 
\[V(2^k) = 2^j*V\left(2^{k-j}\right)+2^{k}*j*b+ c*(2^j-1) \] 
\newline
Nun folgt Verifikation durch Induktion.
Der Induktionsanfang $j=1$ entspricht der Rekursionsgleichung
\[V(n) = 2* V\left(\frac{n}{2}\right)+ b*n +c \]
bzw.
 \[V(2^k) = 2* V\left(2^{k-1}\right)+  2^k*b +c \] 
 Nun folgt Schritt $j \rightarrow j+1$
\[V(2^k) = 2^j*V\left(2^{k-j}\right)+2^{k}*j*b+ c*(2^j-1) \] (Induktionsannahme)
\[V(2^k) = 2^j*\left(2* V\left(2^{k-1-j}\right)+  2^{k-j}*b +c\right)+2^{k}*j*b+ c*(2^j-1) \] 
\[V(2^k) = 2^{j+1}*V\left(2^{k-1-j}\right)+2^{k}*b +2^j*c+2^k*j*b+2^j*c-c \] 
\[V(2^k) = 2^{j+1}*V\left(2^{k-1-j}\right)+(1+j)*2^k*b-(2^{j+1}*c+c)\] 
\newline
für j = k
\[V(2^k) = 2^k*V\left(2^{k-k}\right)+2^{k}*k*b+ c*(2^k-1) \] 
\[V(2^k) = 2^k*V\left(1\right)+2^{k}*k*b+ c*2^k-c \] 
\[V(2^k) = 2^k+2^{k}*(k*b+ c)-c \]  da V(1) = 1
\newline mit $n=2^k$
\[V(n) = n+n*(log_2n*b+ c)-c \] 
\newline
\begin{center}
\begin{tabular}[h]{l|c|c}
n & $2* V\left(\frac{n}{2}\right)+ b*n +c$ & $ n+n*(log_2n*b+ c)-c$ \\
\hline
1 & 1 & 1 \\
\hline
2 & $2+2*b+c$ & $2+2*b+c$ \\
\hline
4 & $4+8*b+3*c$ & $4+8*b+3*c$ \\
\end{tabular}
\end{center}
\end{document}