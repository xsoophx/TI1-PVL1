\documentclass{article}
\usepackage{blindtext}
\usepackage[T1]{fontenc}
\usepackage[utf8]{inputenc}

\title{PVL 1 - theoretische Informatik}
\author{Sophia Köhler}
\date{16.10.2020}

\begin{document}

\maketitle

\section{}
Induktion besteht aus:
\begin{enumerate}
  \item Induktionsanfang.
  \item Induktionsschritt.
  	\begin{enumerate}
	  \item Induktionsbehauptung.
  	\item Induktionsbeweis.
  	\end{enumerate}
\end{enumerate}
\subsection{}
Induktion von \[ \sum_{k=1}^{n}{k} = \frac{(n+1)*n}{2} \]
\subsubsection{Induktionsanfang}

für $n=1$
\[ \sum_{k=1}^{1}{k} = \frac{(1+1)*1}{2} \]
\[ 1 = 1 \]
\newline
für $n=2$
\[ \sum_{k=1}^{2}{k} = \frac{(2+1)*2}{2} \]
\[ 1+2 = 3 \]
\newline
\clearpage
für $n=3$
\[ \sum_{k=1}^{3}{k} = \frac{(3+1)*3}{2} \]
\[ 1+2+3 = 6 \]
\newline
Induktionsanfang stimmt zunächst.
\subsubsection{Induktionsschritt}
\paragraph{Induktionsvorraussetzung}\mbox{}

Vorschrift gilt für: $k=n$ also \[1+2+3+n = \frac{(n+1)*n}{2} \]

\paragraph{Induktionsbehauptung}\mbox{}

Vorschrift gilt auch für: $k=n+1$ also \[1+2+3+...+(n+1) = \frac{(n+2)*(n+1)}{2} \]

\paragraph{Induktionsbeweis}\mbox{}

Aus der Induktionsbehauptung muss nun die Vorraussetzung folgen.

es sollte also gelten: \[1+2+3+...+n+(n+1) = \frac{(n+1)*(n)}{2}+(n+1) = \frac{(n+2)*(n+1)}{2}\]
\[1+2+3+...+n+(n+1) = \frac{(n+1)*(n)}{2}+(n+1) \]
\[1+2+3+...+n+(n+1) = \frac{n^2+n}{2}+ \frac{2*n+2}{2} \]
\[1+2+3+...+n+(n+1) = \frac{n^2+3*n+2}{2}\]
pq Formel für das Lösen der rechten Seite: \[n_{1,2} = -\left(\frac{p}{2}\right) \pm \sqrt{ \left(\frac{p}{2}\right)^{2}-q} \]
 $n_1=-1$ und  $n_2=-2$
 \newline
nun folgt Umformung in Nullstellenform:
\[\frac{(n+2)*(n+1)}{2} \]
\newline
Somit ist gezeigt, dass Induktionsbehauptung aus der Induktionsvorraussetzung folgt.
\end{document}